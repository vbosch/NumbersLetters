%
%  NumGenerator
%
%  Created by vbosch on 2011-04-18.
%  Copyright (c) 2011 McBosch. All rights reserved.
%
\documentclass[a4paper,10pt,titlepage]{article}

% Use utf-8 encoding for foreign characters
\usepackage[utf8]{inputenc}

% Setup for fullpage use
\usepackage{fullpage}
\usepackage{titlepic}
\usepackage{float}
\usepackage{color}

% Uncomment some of the following if you use the features
%
% Running Headers and footers
%\usepackage{fancyhdr}

% Multipart figures
%\usepackage{subfigure}

% More symbols
\usepackage{amsmath}
%\usepackage{amssymb}
%\usepackage{latexsym}

% Surround parts of graphics with box
\usepackage{boxedminipage}

% Package for including code in the document
\usepackage{listings}

% If you want to generate a toc for each chapter (use with book)
\usepackage{minitoc}

% This is now the recommended way for checking for PDFLaTeX:
\usepackage{ifpdf}

%\newif\ifpdf
%\ifx\pdfoutput\undefined
%\pdffalse % we are not running PDFLaTeX
%\else
%\pdfoutput=1 % we are running PDFLaTeX
%\pdftrue
%\fi

\ifpdf
\usepackage[pdftex]{graphicx}
\else
\usepackage{graphicx}
\fi
\title{ACV \\ Course Work - Implementation of an AI Component for the Number Sequence Problem}
\titlepic{\includegraphics[width=250pt]{cover.jpg}}
\author{Vicente Bosch Campos \dag \\
\textcolor{blue}{\texttt{viboscam@posgrado.upv.es}}}

\begin{document}

\ifpdf
\DeclareGraphicsExtensions{.pdf, .jpg, .tif}
\else
\DeclareGraphicsExtensions{.eps, .jpg}
\fi

\maketitle

\tableofcontents

\listoffigures

\newpage

\section{Introduction}
\subsection{Context}

\par As part of the ACV course project we will implement a genetic algorithm in order to resolve the Numbers problem of the Numbers and Letters game. The goal is to obtain a combination of arithmetic operations over a set of 6 integer numbers in order to obtain as a result of those operations another integer number.

\par The rules are as follows:
\begin{itemize}
	\item The six selected numbers must be chosen randomly from: 1,2,3,4,5,6,7,8,9,10,25,50,75 \& 100 taking into account that the probability to choose one of the numbers between 1 and 10 must be double than choosing the rest of the numbers.
	\item The target number must be chosen randomly from the range 101..999.
	\item Operations allowed to combine numbers are: sum, subtraction, multiplication and division. ( division can only be applied if it is exact)
	\item Operations must be performed with the selected numbers or with the numbers obtained by operating on them.
	\item Results of operations must always be integer and positive.
	\item If the exact number can not be obtained, the best result will be the one closest to it.
\end{itemize}

\subsection{Scope}

\par In order to implement this AI component to resolve the number problem we will develop components in order to cover the following user cases:

\begin{itemize}
	\item Evaluate a solution provided by a user against a target value.
	\item Obtain a best solution through running the genetic algorithm resolver. 
	\item Programmer will be allowed to indicate the minimum and maximum number of individuals in the population
	\item The genetic algorithm will be time constrained by the time given for the test to be performed by the human players.
\end{itemize}

\par Experimental evaluation will be performed by testing the specific problems indicated in the project description. 

\section{Development}
\par For our implementation of the sequence number problem we have implemented the algorithm in Ruby (v1.9.2) and the following libraries where used: 
\begin{description}
	\item[RMagick (2.13.1):] RMagick is a wrapper library for the Image Magick software. RMagick has been used in our implementation to load the generated operation graphs.  
	\item[Awesome Print (0.3.1):] Used to perform pre-formatted human readable print outs of the objects.
	\item[rgl (0.4.0):] Was used in order to generate the operation graph drawing.
	\item[Benchmark:] Standard library object was used to measure the time taken to perform operations.
	\item[Timeout:] Standard library object was used in order to enforce the time constraint on the execution of the genetic algorithm.
\end{description}

\subsection{Program Structure}

\par Next we will detail the main software components and characteristics for the sequence number resolver. The implementation can be divided in the following main software components:
\begin{description}
	\item[Application logger:] Coded in \textit{./lib/application\_logger.rb} is a global logger object implemented through the singleton and logger classes to allow the creation of logs easily from any object to a centralized file.
	\item[Number Problem Generator:] Present in \textit{./lib/number\_problem\_generator.rb} is the component in charge of loading a problem from a file or generating one randomly.
	\item[Operation Graph:] Class coded in \textit{./lib/operation\_graph.rb}. It represents a population individual for our sequence number problem. The operation graph is stored internally as an actual graph of nodes that allow breed and mutation operations to be evaluated in constant time. Each node contains:
	\begin{itemize}
		\item A list of the numbers consumed beneath it and the current node type.
		\item Links to the root, its father and its left and right children.
		\item List of all of the numbers consumed and unconsumed in the whole of the operation graph.
	\end{itemize}
	The operation graph object allows the following methods:
	\begin{itemize}
		\item from\_string - Allows an operation graph to be constructed from a string containing the operations to perform in polish notation. 
		\item initialize - Performs a random initialization of the operation graph by selecting a random set of the possible values and generating a tree by subdividing the selected numbers into different groups at each node and combining them with a randomly selected operator.  
		\item level= - Sets the level of the node inside the hierarchy ( root node has level 0) and sets recursively the level on its child nodes.
		\item unconsumedNumbers - Returns the list of unused numbers in the whole of the operation graph.
		\item operatorNodes - Returns the full list of nodes that act as operation nodes inside the graph.
		\item valueNodes - Returns the full list of nodes that act as value nodes inside the graph.
		\item value - Returns the value of a node.
		\item value= - Sets the value of a node and updates the consumedNumbers list recursively in all of its parent nodes.
		\item result - Calculates the result of the operation graph
		\item score - Returns the distance ,as a positive integer value, between the current result of the operation graph and the expected target value.
		\item clone - Performs a deep copy of the operation graph.
		\item transform2Child - Transforms an specific node into  value type node.
		\item transform2Op - Transforms an specific node into operator type node.
		\item changeConsumedNumbers - Modifies the consumedNumbers list of the node and recursively updates the parent nodes.
		\item addConsumedNumber - Adds a new value to the consumedNumbers list of the node and recursively updates the parent nodes.
		\item deleteConsumedNumbers - Deletes the consumedNumbers list of the node and recursively updates the parent nodes.
		\item draw - Using the rgl library it draws a visual representation of the operation graph and shows it to the user.
		\item mutate! - Selects at random a method terminated in *Mutation, by means of the reflection methods in ruby, and executes it over the current operation graph.
		\item cloneMutant - Returns a cloned an operation graph on which mutate! has been executed on. 
		\item operatorSwitchMutation - Selects one of the operator nodes at random and changes the operation performed in it. 
		\item numberSwitchMutation - Selects two value nodes in the operation graph and swaps them. 
		\item unusedNumberSwitchMutation - Changes a value used in a node of the operation graph by an unused number. 
		\item unusedNumberMutation - Selects an unused number from the original set and combines it with a current value node to create a subtree composed of a new operation node, the existing value node and a new value node for the unconsumed number hanging in the operation graph in the same position as the existing value node prior to the mutation. 
		\item deleteOperationMutation - Deletes all nodes below an operation node updating the value node list,operator node list and the unconsumed numbers list. 
		\item deleteSubTreeOperators - Deletes all operator nodes in a sub tree beneath a selected node. Used by the deleteOperationMutation. 
		\item breed - Combines two operation graphs into a new child. If the base graphs do not share any consumed number it just randomly selects two cross points (one at each graph) and combines them. If the base graphs do share consumed numbers it then compiles a list of pairs of base nodes that can be combined together  and selects one of those pairs to combine them so that the child graph will be compliant with the base rules. If there exists no possible combination the operation returns a cloned mutant of the mother.
		\item setChildren - Sets the children nodes of an specific node. 
		\item getTypeNodes - Returns a list with all of the nodes of an specific type in a subtree beneath a node.
	\end{itemize}
	\item[Genetic Population:] Main object of the genetic algorithm implementation present in \textit{./lib/genetic\_population.rb}. The genetic algorithm implementation has the following base methods:
	\begin{itemize}
		\item initialize - Creates an initial population of a random number of individuals between the indicated minimum and maximum population.
		\item reducePopulation - This method reduces the population until it reaches the upper bound indicated by the user. The reduction is performed by sorting the children by score value and eliminating the worst individuals until the limit is reached.
		\item updateAverage - Calculates the average score of the current population. 
		\item evaluatePopulation - Reviews all current individuals, it sorts them by score value, stores the best individual and eliminates any individuals with a negative score.
		\item recoverPopulation - If the current population is above the minimal population people indicated by the user it recovers it by generating randomly new individuals until the limit is reached.
		\item draw - Draws the operation graph of all current individuals.
		\item scores - Returns a histogram of the number of operation graphs per score value. 
		\item draw\_best\_history - Plots the score evolution of the best result in the population. 
		\item draw\_avg\_history - Plots the score evolution of the average result of the population.
		\item draw\_population\_history - Plots the population count through history.
		\item draw\_scores - Draws the score histogram of the current population. 
		\item actOnIndividual - Performs a random action on an individual. Action can be:
		\begin{itemize}
			\item Mutate
			\item Clone
			\item Clone a mutant
			\item Breed
		\end{itemize}
		\item selectIndividuals - Returns a list of the current individuals that have a score between the current best and the current best plus a random \% from 20 to 50 of the target value. 
		\item evolve - Is the main evolution loop. The evolve method has been implemented so it can be time constrained by the user. If the algorithm obtains an exact solution it will stop immediately and provide it in case it does not it will continue to iterate until the indicated computation time is over at which point it will return the best solution obtained. 
	\end{itemize}
\end{description}	
		


\subsection{Command Output}

\par The generated code will present us with the following outputs:
\begin{description}
	\item[Solution reached before time constraint is reached:] In this case the program will output the following messages:
		{\footnotesize\begin{verbatim}
		I, [2011-04-20T11:44:52.994423 #4635]  INFO -- : LOGGER CREATED
		Cycles performed: 19 
		Target value: 718
		Smallest from target: 0
		\end{verbatim}}
	\item[Solution not reached before time constraint is reached:] For the other case the program will present the messages:
	{\footnotesize\begin{verbatim}
	I, [2011-04-20T12:10:29.158400 #5150]  INFO -- : LOGGER CREATED
	I, [2011-04-20T12:10:34.168199 #5150]  INFO -- : Termination due to time constraint
	Cycles performed: 131 
	Target value: 919
	Smallest from target: 1
	\end{verbatim}}
\end{description}

\par The component can present us with the operation graph of the best solution and a graph representing the best solution score's evolution through time.

\par \begin{figure}[H]
	\centerline{%
	\includegraphics[scale=0.5]{./plots/best_evolution.eps}
	}
	\caption[Sequence number resolution: Best score evolution through time]{Plot shows the best score, as a distance between the best individuals result and the target distance, as a function of the evolution cycles performed.}
\end{figure}

\par \begin{figure}[H]
	\centerline{%
	\includegraphics[scale=0.5]{./plots/operation_graph.png}
	}
	\caption[Sequence number resolution: Sample operation graph visualization]{Diagram shows a visualization of an operation graph where besides the operators the object id has been included so that the nodes are unique.}
\end{figure}


\section{Experimental Results}

\par We have evaluated the performance of our implementation of the genetic algorithm resolver we will experiment over a base set of problems:

\begin{table}[H] 
\caption{Provided experiments set} %title of the table
\centering 
\begin{tabular}{c c c c c c c c}
			
\multicolumn{8}{c}{\textbf{Provided Experiments Set}} \\
\hline

\multicolumn{1}{c}{Experiment Id}
&\multicolumn{6}{c}{Initial Values}
&\multicolumn{1}{c}{Target Value} \\

\cline{2-7}

F1 & 4 & 10 & 7 & 9 & 2 & 25 & 232 \\
F2 & 10 & 2 & 9 & 5 & 7 & 100 &	298 \\
F3 & 2 & 75 & 9 & 6 & 100 & 8 &	474 \\
F4 & 3 & 10 & 7 & 6 & 75 & 10 &	381 \\
F5 & 7 & 3 & 50 & 25 & 6 & 100 & 741 \\
F6 & 2 & 4 & 75 & 4 & 50 & 2 & 502 \\
F7 & 8 & 25 & 10 & 2 & 9 & 4 & 549 \\
F8 & 7 & 10 & 3 & 2 & 1 & 25 & 223 \\
F9 & 7 & 4 & 100 & 5 & 9 & 10 & 156 \\
F10 & 2 & 10 & 7 & 50 & 100 & 10 & 259 \\
D1 & 9 & 6 & 75 & 7 & 2 & 50 & 264 \\
D2 & 5 & 3 & 9 & 50 & 4 & 5 & 458 \\
D3 & 1 & 10 & 3 & 3 & 75 & 4 & 322 \\
D4 & 100 & 2 & 6 & 4 & 25 & 6 & 305 \\
D5 & 2 & 1 & 10 & 10 & 7 & 100 & 274 \\
D6 & 1 & 100 & 4 & 50 & 3 & 4 & 661 \\
D7 & 10 & 8 & 75 & 2 & 100 & 4 & 431 \\
D8 & 100 & 75 & 1 & 8 & 3 & 75 & 511 \\
D9 & 100 & 3 & 2 & 25 & 3 & 9 &	407 \\
D10 & 5 & 50 & 1 & 8 & 75 & 8 &	713 

\end{tabular} 
\label{tab:dist_result} 
\end{table}

\par We will perform two sets of experimentation:
\begin{itemize}
	\item Unconstrained: We will allow the random tree builder to use as many of the selected numbers as it requires.
	\item Constrained: We will only allow the tree builder to select one of the selected numbers, hence it will a very basic solution.
\end{itemize}

\par For each problem configuration described in the table above we will perform 50 experiments with a time constraint of 10 seconds and record for each:
\begin{itemize}
	\item Number of times exact solution is reached 
	\item Average number of cycles for exact resolution
	\item Number of times time constrained solution is given
	\item Average number of cycles for time constrained resolution
	\item Average distance from target Average number of cycles for exact resolution
	\item Best distance from target Average number of cycles for exact resolution
\end{itemize}

\subsection{Unconstrained Initialization}

\par The results for the unconstrained generation are as follows:

\begin{table}[H] 
\caption{Unconstrained Tree Generation Experiments Statistics} %title of the table
\centering 
\begin{tabular}{c @{\extracolsep{12pt}} c c @{\extracolsep{12pt}} c c c c }

\multicolumn{7}{c}{\textbf{Experiments Statistics}} \\
\hline
Experiment Id & \multicolumn{2}{c}{Exact Resolution} & \multicolumn{4}{c}{Time Constrained Resolution}\\
\cline{2-3}\cline{4-7}
 & Count & Avg. Cycles & Count & Avg. Cycles & Avg. Score & Best Score\\
\cline{2-3}\cline{4-7}
F1 & 50.0 & 13.8 & 0.0 & 0.0 & 0.0 & -1 \\
F2 & 28.0 & 59.678 & 22.0 & 221.818 & 1.545 & 1 \\
F3 & 25.0 & 83.28 & 25.0 & 194.96 & 1.08 & 1 \\
F4 & 23.0 & 50.043 & 27.0 & 206.629 & 1.296 & 1 \\
F5 & 38.0 & 58.657 & 12.0 & 192.916 & 1.0 & 1 \\
F6 & 39.0 & 41.487 & 11.0 & 191.636 & 2.0 & 2 \\
F7 & 14.0 & 71.857 & 36.0 & 202.361 & 1.0 & 1 \\
F8 & 48.0 & 49.875 & 2.0 & 224.5 & 1.0 & 1 \\
F9 & 50.0 & 13.76 & 0.0 & 0.0 & 0.0 & -1 \\
F10 & 40.0 & 53.575 & 10.0 & 226.2 & 1.3 & 1 \\
D1 & 43.0 & 43.674 & 7.0 & 231.142 & 1.142 & 1 \\
D2 & 32.0 & 32.031 & 18.0 & 235.888 & 1.0 & 1 \\
D3 & 43.0 & 49.441 & 7.0 & 206.285 & 4.571 & 1 \\
D4 & 7.0 & 90.857 & 43.0 & 228.651 & 1.0 & 1 \\
D5 & 2.0 & 83.0 & 48.0 & 231.979 & 1.375 & 1 \\
D6 & 8.0 & 94.25 & 42.0 & 230.452 & 2.166 & 1 \\
D7 & 24.0 & 99.875 & 26.0 & 227.192 & 1.0 & 1 \\
D8 & 23.0 & 71.826 & 27.0 & 225.777 & 3.925 & 2 \\
D9 & 21.0 & 75.095 & 29.0 & 232.379 & 1.0 & 1 \\
D10 & 35.0 & 84.914 & 15.0 & 229.866 & 1.0 & 1 \\

\end{tabular} 
\label{tab:uncon_result} 
\end{table}


\subsection{Constrained Initialization}
\begin{table}[H]
\caption{Constrained Tree Generation Experiments Statistics} %title of the table
\centering 
\begin{tabular}{c @{\extracolsep{12pt}} c c @{\extracolsep{12pt}} c c c c }
			
\multicolumn{7}{c}{\textbf{Experiments Statistics}} \\
\hline
Experiment Id & \multicolumn{2}{c}{Exact Resolution} & \multicolumn{4}{c}{Time Constrained Resolution}\\
\cline{2-3}\cline{4-7}
 & Count & Avg. Cycles & Count & Avg. Cycles & Avg. Score & Best Score\\
\cline{2-3}\cline{4-7}
F1 & 38.0 & 79.578 & 12.0 & 289.916 & 1.0 & 1 \\
F2 & 24.0 & 82.833 & 26.0 & 308.884 & 1.692 & 1 \\
F3 & 38.0 & 79.263 & 12.0 & 283.25 & 2.5 & 1 \\
F4 & 10.0 & 127.1 & 40.0 & 310.6 & 1.95 & 1 \\
F5 & 9.0 & 74.888 & 41.0 & 286.609 & 1.975 & 1 \\
F6 & 14.0 & 68.714 & 36.0 & 313.583 & 2.0 & 2 \\
F7 & 23.0 & 140.130 & 27.0 & 306.148 & 1.0 & 1 \\
F8 & 18.0 & 77.222 & 32.0 & 299.406 & 1.0 & 1 \\
F9 & 50.0 & 27.18 & 0.0 & 0.0 & 0.0 & -1 \\
F10 & 36.0 & 57.0 & 14.0 & 314.0714 & 1.0 & 1 \\
D1 & 43.0 & 70.861 & 7.0 & 281.285 & 1.571 & 1 \\
D2 & 8.0 & 21.5 & 42.0 & 283.547 & 1.5 & 1 \\
D3 & 18.0 & 82.055 & 32.0 & 306.0 & 2.468 & 1 \\
D4 & 3.0 & 176.0 & 47.0 & 311.425 & 1.085 & 1 \\
D5 & 2.0 & 170.5 & 48.0 & 317.666 & 2.166 & 1 \\
D6 & 2.0 & 42.5 & 48.0 & 310.875 & 1.166 & 1 \\
D7 & 9.0 & 187.777 & 41.0 & 288.439 & 1.731 & 1 \\
D8 & 15.0 & 157.533 & 35.0 & 298.542 & 4.028 & 1 \\
D9 & 6.0 & 91.0 & 44.0 & 315.091 & 1.023 & 1 \\
D10 & 14.0 & 110.214 & 36.0 & 305.555 & 1.0 & 1 \\
\end{tabular} 
\label{tab:con_result} 
\end{table}

\subsection{Conclusions}

\par As we have seen in the experimentation above:
\begin{itemize}
	\item The ratio of exact solutions is significantly worse for the problems tagged as hard.
	\item The unconstrained initialization of the operation graphs yield  better results.
	\item In both constrained and unconstrained initializations the solutions provided in 10 seconds are very close from the target either an exact solution is reached or at an average distance of 2 from the target value.
\end{itemize} 

\end{document}
